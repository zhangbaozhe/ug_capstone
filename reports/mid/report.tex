\documentclass[letterpaper,journal]{IEEEtran}
\usepackage{amsmath,amsfonts}
\usepackage{array}
\usepackage{textcomp}
\usepackage{stfloats}
\usepackage{url}
\usepackage{verbatim}
\usepackage{graphicx}
\usepackage{cite}
\usepackage{xcolor}
\usepackage{subcaption}
\usepackage{mathtools}  
\usepackage{amssymb}
\usepackage{tabulary}
\usepackage{booktabs}
\usepackage[ruled,linesnumbered]{algorithm2e}
\usepackage{hyperref}
\usepackage{setspace}

\begin{document}

\title{(TODO: narrowing down) Multi-Robot Path Planning in Dynamic Environments}

\author{Baozhe Zhang
        % <-this % stops a space
\thanks{TODO}% <-this % stops a space
\thanks{TODO}}

% The paper headers
\markboth{ERG4901: Capstone Project -- Mid-Term Report}%
%{Shell \MakeLowercase{\textit{et al.}}: A Sample Article Using IEEEtran.cls for IEEE Journals}
{Zhang: Mid-Term Report}

\IEEEpubid{\copyright~2023 Baozhe Zhang}
% Remember, if you use this you must call \IEEEpubidadjcol in the second
% column for its text to clear the IEEEpubid mark.

\maketitle

\begin{abstract}
TODO
\end{abstract}

\begin{IEEEkeywords}
  TODO
\end{IEEEkeywords}

\section{Introduction}
\IEEEPARstart{T}{odo}

\section{Related Work}

In general, path-planning algorithms for robot navigation can be divided into two major categories: global and local methods.
Global methods focus on finding collision-free paths from the current robot position to a global goal position mostly in static scenes, while local methods tend to reactively avoid both static and dynamic obstacles. 
Path-planning algorithms for multi-robot systems \dots (TODO: cooperative, distributed, heterogeneous)

Traditional global methods such as rapidly exploring random tree (RRT), probabilistic roadmap (PRM), and their variants use sampling in the configuration space to crate collision-free pathss \cite{lavalle2001rapidly}, \cite{karaman2011sampling}, \cite{kavraki1996probabilistic}. 
However, these methods need to search the whole space which may suffer from heavy computational loads especially when the searching space is large.  
Optimization-based method such as model predictive control added with obstacle constraints or with additional environment-related terms \cite{park2009obstacle}, \cite{ji2016path}  can be used to generate collision-free paths. 
However, MPC-related methods need to tradeoff among the length of the time horizon, the computational loads, and optimality. Short time horizon often leads the problem to local optimized solution.

\bibliographystyle{IEEEtran}
\bibliography{report}


\end{document}


